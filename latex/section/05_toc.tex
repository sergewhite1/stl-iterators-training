\section{Об оглавлении}

\noindent
\sbdef{
    Рубрикация (sectioning) --- это разбивка документа на части, главы, параграфы и т.п.}

\bigskip

\noindent
\verb|\tableofcontents| --- формирование и вставка оглавления.

\begin{table}[h]
    \caption{Возможный состав рубрикации}
    \label{sectioning}
    \begin{tabular}{|r|l|p{0.7\textwidth}|}
        \hline
        \textbf{№}        &
        \textbf{Команда}  &
        \textbf{Описание}                                              \\ \hline
        1 & part          & часть                                      \\ \hline
        2 & chapter       & глава(для класса документа article не
                            работает)                                  \\ \hline
        3 & section       & секция (типа параграф, но это слово в
                            команде №6)                                \\ \hline
        4 & subsection    & подсекция                                  \\ \hline
        5 & subsubsection & подподсекция                               \\ \hline
        6 & paragraph     & пункт (типа абзац)                         \\ \hline
        7 & subparagraph  & подпункт                                   \\ \hline
    \end{tabular}
\end{table}

\noindent
"<звёздочные версии команд"> производят вставку сущности без номера и не влияющую на дальнейшую нумерацию.

\bigskip

\noindent
\verb|\addcontentsline{toc}{part}{<имя в оглавлении>}| --- принудительное добавление ненумерованной сущности в оглавление. Например:

\medskip

\noindent
\begin{boxedverbatim}
\section*{Литерура}
\addcontentsline{toc}{part}{Литература}
\end{boxedverbatim}

\clearpage