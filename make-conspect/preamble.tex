\documentclass[a4paper, 12pt]{article}

% Работа с русским языком ======================================================

\usepackage[T2A]{fontenc}             % Кодировка
\usepackage[english, russian]{babel}  % Локализация и переносы
\usepackage[utf8]{inputenc}           % Кодировка исходного текста
\usepackage{indentfirst}              % Красная строка в первом абзаце

% Оформление страницы ==========================================================

% Поля

\usepackage{geometry}
\geometry{top=25mm}
\geometry{bottom=30mm}
\geometry{left=20mm}
\geometry{right=20mm}

\linespread{1.5} % Коэффициент межстрочного интервала

% Колонтитулы

\usepackage{titleps}
\newpagestyle{main}{
	\setheadrule{0.4pt}
	\sethead{left}{center}{right}
	\setfootrule{0.4pt}
	\setfoot{left}{\thepage}{right}
}

\pagestyle{main}

% Работа с картинками ==========================================================
\usepackage{graphicx} % Для вставки рисунков

% Математика ===================================================================

\usepackage{amsmath}
\usepackage{amsfonts}
\usepackage{mathtools}

\newcommand{\deriv}[2]{\frac{\partial #1}{\partial #2}}
\newcommand{\R}{\mathbb R}

\renewcommand{\phi}{\varphi}

\DeclareMathOperator{\Kerr}{Ker}
\DeclareMathOperator{\Ree}{Re}
\DeclareMathOperator{\Imm}{Im}

% Здесь звёздочка перевод нижний индекс под оператор в формуле только на лной строке
\DeclareMathOperator*{\argmax}{argmax}

% Бинарная операция
\newcommand{\percent}{\mathbin{\%}}

\usepackage{amsthm}    % Создание теорем
\theoremstyle{plain}   % {plain, definition, remark}
\newtheorem{theorem}{Теорема}[section]
\newtheorem{corollary}{Следствие}[theorem]

% Звёздочка отменяет нумерацию
\newtheorem*{definition}{Определение}

% ==============================================================================

\newenvironment{myenv}
    {\begin{center}\bfseries}
    {\end{center}}

\usepackage{hyperref}

\usepackage{tikz}
\usetikzlibrary{calc,intersections,through,backgrounds}
\usetikzlibrary{graphs}
\usetikzlibrary{shapes.geometric}
\usepackage{tkz-euclide}

% Листинги

\usepackage{listings}

\lstdefinestyle{mystyle}
{
% backgroundcolor=\color{black!5}, % set backgroundcolor
  basicstyle=\footnotesize,% basic font setting
  breaklines=true,
  frame=single,
  keywordstyle=\color{blue}\textbf,
  language=C++,
  numbers=left
}

\lstset {style=mystyle}