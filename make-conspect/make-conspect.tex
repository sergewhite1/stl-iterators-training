\documentclass[a4paper, 12pt]{article}

% Работа с русским языком ======================================================

\usepackage[T2A]{fontenc}             % Кодировка
\usepackage[english, russian]{babel}  % Локализация и переносы
\usepackage[utf8]{inputenc}           % Кодировка исходного текста
\usepackage{indentfirst}              % Красная строка в первом абзаце

% Оформление страницы ==========================================================

% Поля

\usepackage{geometry}
\geometry{top=25mm}
\geometry{bottom=30mm}
\geometry{left=20mm}
\geometry{right=20mm}

\linespread{1.5} % Коэффициент межстрочного интервала

% Колонтитулы

\usepackage{titleps}
\newpagestyle{main}{
	\setheadrule{0.4pt}
	\sethead{left}{center}{right}
	\setfootrule{0.4pt}
	\setfoot{left}{\thepage}{right}
}

\pagestyle{main}

% Работа с картинками ==========================================================
\usepackage{graphicx} % Для вставки рисунков

% Математика ===================================================================

\usepackage{amsmath}
\usepackage{amsfonts}
\usepackage{mathtools}

\newcommand{\deriv}[2]{\frac{\partial #1}{\partial #2}}
\newcommand{\R}{\mathbb R}

\renewcommand{\phi}{\varphi}

\DeclareMathOperator{\Kerr}{Ker}
\DeclareMathOperator{\Ree}{Re}
\DeclareMathOperator{\Imm}{Im}

% Здесь звёздочка перевод нижний индекс под оператор в формуле только на лной строке
\DeclareMathOperator*{\argmax}{argmax}

% Бинарная операция
\newcommand{\percent}{\mathbin{\%}}

\usepackage{amsthm}    % Создание теорем
\theoremstyle{plain}   % {plain, definition, remark}
\newtheorem{theorem}{Теорема}[section]
\newtheorem{corollary}{Следствие}[theorem]

% Звёздочка отменяет нумерацию
\newtheorem*{definition}{Определение}

% ==============================================================================

\newenvironment{myenv}
    {\begin{center}\bfseries}
    {\end{center}}

\usepackage{hyperref}

\usepackage{tikz}
\usetikzlibrary{calc,intersections,through,backgrounds}
\usetikzlibrary{graphs}
\usetikzlibrary{shapes.geometric}
\usepackage{tkz-euclide}

% Листинги

\usepackage{listings}
\usepackage{xcolor}

\definecolor{mygreen}{rgb}{0.0, 0.15, 0.05}

\lstdefinestyle{mystyle}
{
% backgroundcolor=\color{black!5}, % set backgroundcolor
  basicstyle=\footnotesize,% basic font setting
  breaklines=true,
  commentstyle=\color{mygreen},
  frame=single,
  keywordstyle=\color{blue}\textbf,
  language=C++,
  numbers=left,
  extendedchars=\true
}

\lstset {style=mystyle}

% Для вставки текстового файла с переносом строк
\usepackage{verbatim}

% Для задания межстрочного интервала
\usepackage{setspace}

\begin{document}
Конспект по системе сборки \textbf{make}

\vspace{0.2cm}

Краткое неточное определение:

\textbf{make} - это система сборки проектов в ОС Linux

Более полное определение из Википедии:

\textbf{make} - утилита, автоматизирующая процесс преобразования файлов из одной формы в другую. Чаще всего это компиляция исходного кода в объектные файлы и последующая компоновка в исполняемые файлы или библиотеки.

\vspace{0.2cm}

Данный конспект рассматривает make применительно к сборке C/C++ проектов на ОС Linux с использование компилятора gcc

Поэтому сперва рассмотрим сборку С/C++ проектов

\vspace{0.2cm}

\begin{tikzpicture}
	% типа рисование по клеточкам
%	\draw [help lines] (0, 0) grid (16, 12);

	\node (c_file_0) at ( 0, 11)  [shape=rectangle, draw, minimum width=2cm, minimum height=2cm]{.c};
	\node (c_file_1) at ( 0,  8)  [shape=rectangle, draw, minimum width=2cm, minimum height=2cm]{.c};
	\node (c_file_2) at ( 0,  5)  [shape=rectangle, draw, minimum width=2cm, minimum height=2cm]{.c};
	\node (preproc)  at ( 3,  8)  [shape=rectangle, draw, minimum width=2cm, minimum height=8cm, text width=1.7cm]{Preproc.\\\#include\\\#ifndef\\\#define\\...};
	\node (compiler) at ( 6,  8)  [shape=rectangle, draw, minimum width=2cm, minimum height=8cm]{Compiler};
	\node (o_file_0) at ( 9, 11)  [shape=rectangle, draw, minimum width=2cm, minimum height=2cm, text width=1cm]{.o\\(.obj)};
	\node (o_file_1) at ( 9,  8)  [shape=rectangle, draw, minimum width=2cm, minimum height=2cm, text width=1cm]{.o\\(.obj)};
	\node (o_file_2) at ( 9,  5)  [shape=rectangle, draw, minimum width=2cm, minimum height=2cm, text width=1cm]{.o\\(.obj)};
	\node (linker)   at (12,  8)  [shape=rectangle, draw, minimum width=2cm, minimum height=8cm]{Linker};
	\node (exe)      at (15,  8)  [shape=rectangle, draw, minimum width=2cm, minimum height=2cm, text width=1cm]{exe\\(.exe)};

	\draw[->] (c_file_0.east) -- ++(right:1);
	\draw[->] (c_file_1.east) -- ++(right:1);
	\draw[->] (c_file_2.east) -- ++(right:1);

	\draw[->] ( 4, 11) -- (5,11);
	\draw[->] ( 4,  8) -- (5, 8);
	\draw[->] ( 4,  5) -- (5, 5);

	\draw[->] ( 7, 11) -- (8,11);
	\draw[->] ( 7,  8) -- (8, 8);
	\draw[->] ( 7,  5) -- (8, 5);

	\draw[->] (10, 11) -- (11, 11);
	\draw[->] (10,  8) -- (11,  8);
	\draw[->] (10,  5) -- (11,  5);

	\draw[->] (13,  8) -- (14,  8);

	\node (or_1)       at (15, 6.5) [shape=rectangle]{or};

	\node (static_lib) at (15,   5) [shape=rectangle,draw, minimum width=2cm, minimum height=2cm, text width=1cm]{.a\\(.lib)};

	\node (or_2)       at (15, 3.5) [shape=rectangle]{or};

	\node (dynamic_lb) at (15,   2) [shape=rectangle,draw, minimum width=2cm, minimum height=2cm, text width=1cm]{.so\\(.dll)};

\end{tikzpicture}

Типы выходных файлов

\begin{tabular}{|l|l|l|}
	\hline
	\textbf{Тип выходного файла} & \textbf{Расширение в ОС Linux} & \textbf{Расширение в ОС Windows} \\
	\hline
	Исполняемый файл & без расширения & .exe (от executable) \\
	\hline
	Статическая библиотека & .a (от archive) & .lib (от library) \\
	\hline
	Динамическая библиотека & .so (от shared object) & .dll (от dynamic link library) \\
	\hline
\end{tabular}

\clearpage

Далее для справки приведены некоторые команды gcc.

Для начала рассмотрим получение результата работы препроцессора. Рассмотрим такой код (файл main.cpp):


\begin{lstlisting}
#include <iostream>

#define CONST_A 100

int main()
{
  std::cout << "CONST_A=" << CONST_A << std::endl;
  return 0;
}
\end{lstlisting}


Следующая команда позволяет получить выход препроцессора:

\begin{lstlisting}[language=csh]
gcc -E main.cpp > main.cpp.i
\end{lstlisting}

В результате получаем большой файл. Вот его последние строки:

\begin{lstlisting}[firstnumber=28636]
# 2 "main.cpp" 2




# 5 "main.cpp"
int main()
{
	std::cout << "CONST_A=" << 100 << std::endl;
	return 0;
}
\end{lstlisting}

Компиляция исходного файла в объектный происходит следующей командой:

\begin{lstlisting}[language=csh]
gcc -c main.cpp
\end{lstlisting}

В результате получится файл main.o Имя выходного файлы можно задавать явно:

\begin{lstlisting}[language=csh]
gcc -c main.cpp -o main.o
\end{lstlisting}

Для просмотра таблицы символов в объектом файле можно использовать утилиты \textbf{nm} или \textbf{readelf}:

\begin{lstlisting}[language=csh, numbers=none]
$ nm -C main.o
U _GLOBAL_OFFSET_TABLE_
0000000000000091 t _GLOBAL__sub_I_main
0000000000000044 t __static_initialization_and_destruction_0(int, int)
U std::ostream::operator<<(std::ostream& (*)(std::ostream&))
U std::ostream::operator<<(int)
U std::ios_base::Init::Init()
U std::ios_base::Init::~Init()
U std::cout
U std::basic_ostream<char, std::char_traits<char> >& std::endl<char, std::char_traits<char> >(std::basic_ostream<char, std::char_traits<char> >&)
0000000000000000 r std::piecewise_construct
0000000000000000 b std::__ioinit
U std::basic_ostream<char, std::char_traits<char> >& std::operator<< <std::char_traits<char> >(std::basic_ostream<char, std::char_traits<char> >&, char const*)
U __cxa_atexit
U __dso_handle
0000000000000000 T main
\end{lstlisting}

Здесь, на 16-ой строчке мы видим функцию main. Ключ -С означает снятие декорирования с идентификаторов.

Вот пример с утилитой \textbf{readlef}:

\begin{lstlisting}[language=csh, numbers= none]
serge@serge:~/tmp2$ readelf -sW main.o

Symbol table '.symtab' contains 26 entries:
Num:    Value          Size Type    Bind   Vis      Ndx Name
0: 0000000000000000     0 NOTYPE  LOCAL  DEFAULT  UND
1: 0000000000000000     0 FILE    LOCAL  DEFAULT  ABS main.cpp
2: 0000000000000000     0 SECTION LOCAL  DEFAULT    1
3: 0000000000000000     0 SECTION LOCAL  DEFAULT    3
4: 0000000000000000     0 SECTION LOCAL  DEFAULT    4
5: 0000000000000000     0 SECTION LOCAL  DEFAULT    5
6: 0000000000000000     1 OBJECT  LOCAL  DEFAULT    5 _ZStL19piecewise_construct
7: 0000000000000000     1 OBJECT  LOCAL  DEFAULT    4 _ZStL8__ioinit
8: 0000000000000044    77 FUNC    LOCAL  DEFAULT    1 _Z41__static_initialization_and_destruction_0ii
9: 0000000000000091    25 FUNC    LOCAL  DEFAULT    1 _GLOBAL__sub_I_main
10: 0000000000000000     0 SECTION LOCAL  DEFAULT    6
11: 0000000000000000     0 SECTION LOCAL  DEFAULT    9
12: 0000000000000000     0 SECTION LOCAL  DEFAULT   10
13: 0000000000000000     0 SECTION LOCAL  DEFAULT   11
14: 0000000000000000     0 SECTION LOCAL  DEFAULT    8
15: 0000000000000000    68 FUNC    GLOBAL DEFAULT    1 main
16: 0000000000000000     0 NOTYPE  GLOBAL DEFAULT  UND _ZSt4cout
17: 0000000000000000     0 NOTYPE  GLOBAL DEFAULT  UND _GLOBAL_OFFSET_TABLE_
18: 0000000000000000     0 NOTYPE  GLOBAL DEFAULT  UND _ZStlsISt11char_traitsIcEERSt13basic_ostreamIcT_ES5_PKc
19: 0000000000000000     0 NOTYPE  GLOBAL DEFAULT  UND _ZNSolsEi
20: 0000000000000000     0 NOTYPE  GLOBAL DEFAULT  UND _ZSt4endlIcSt11char_traitsIcEERSt13basic_ostreamIT_T0_ES6_
21: 0000000000000000     0 NOTYPE  GLOBAL DEFAULT  UND _ZNSolsEPFRSoS_E
22: 0000000000000000     0 NOTYPE  GLOBAL DEFAULT  UND _ZNSt8ios_base4InitC1Ev
23: 0000000000000000     0 NOTYPE  GLOBAL HIDDEN   UND __dso_handle
24: 0000000000000000     0 NOTYPE  GLOBAL DEFAULT  UND _ZNSt8ios_base4InitD1Ev
25: 0000000000000000     0 NOTYPE  GLOBAL DEFAULT  UND __cxa_atexit
\end{lstlisting}

Здесь ключ -s означает печатать таблицу символов, а ключ W - не ограничивать длину строки (по умолчанию длина строки ограничена 80 символами).

Здесь мы видим декорированные имена идентификаторов. Чтобы снять декорирование можно использовать утилиту \textbf{c++filt}:

\begin{lstlisting}[language=csh]
$ c++filt _ZStlsISt11char_traitsIcEERSt13basic_ostreamIcT_ES5_PKc
std::basic_ostream<char, std::char_traits<char> >& std::operator<< <std::char_traits<char> >(std::basic_ostream<char, std::char_traits<char> >&, char const*)
\end{lstlisting}

В качестве примера рассмотрим программу решения квадратного уравнения. О квадратном уравнении см. Приложение-1

Программу назовем ques\_exe\footnote{ques - от Quadratic Equation Solver; exe - executable; читается как [квэс].}. Это консольное приложение, которое запрашивает у пользователя коэффициенты уравнения $a$, $b$ и $c$. В результате выводит дискриминант. В случае не отрицательного дискриминанта выводятся корни. Программа работает только в области вещественных чисел. Также сделаем юнит-тесты.  Это программа ques\_ut. Для начала все файлы с исходным кодом разместим в одном каталоге. У нас получается следующая структура файлов (на "рисунке" показаны зависимости):

\texttt{\setstretch{0}\verbatiminput{ques_01.txt}}

Для начала расположим все файлы в одном каталоге. Далее представлены листинги.

\lstinputlisting[caption={ques.h}]{projects/ques_01/ques.h}
\lstinputlisting[caption={ques.c}]{projects/ques_01/ques.c}
\lstinputlisting[caption={main.c}]{projects/ques_01/main.c}
\lstinputlisting[caption={ut.h}]{projects/ques_01/ut.h}
\lstinputlisting[caption={ut.h}]{projects/ques_01/ut.c}
\lstinputlisting[caption={ques\_ut.c}]{projects/ques_01/ques_ut.c}

\textbf{Шаг 1. Сборка обычным bash скриптом}

Артефакты сборки будут находится в одном каталоге с исходными файлами.

\lstinputlisting[caption={build.sh}, language=sh]{projects/ques_01/build.sh}

Для очисти будем вызывать скрипт clean.sh:

\lstinputlisting[caption={clean.sh}, language=sh]{projects/ques_01/clear.sh}

В общем схема рабочая. Что хотелось бы улучшить? Каждый раз при вызове скрипта build.sh происходит полная сборка проекта. В дальнейшем при изменении некоторых исходных файлов хотелось бы использовать уже существующие, скомпилированные объектные файлы. То есть компилировать только требуемые исходники и перекомпоновать проект (слинковать заново). С точки зрения производительности компиляция является более тяжеловесной чем линковка, особенно когда компиляция происходит с оптимизацией. Поэтому для ускорения итерации\footnote{здесь под итерацией понимается: изменение файла с исходным кодом, сборка проекта и его запуск} хочется осуществить такой подход.

\textbf{Шаг 2. make - основы}

\clearpage

\textbf{Шпаргалка}

\# В чистом проекте (без сборки) \\
\texttt{make -pn} > db.txt \\
\texttt{-p} - print-data-base \\
\texttt{-n} - just-print, dry-run, recon

\clearpage

\textbf{Приложение-1}

Квадратным уравнением называется уравнение вида:

\begin{equation}
  ax^2 + bx + c = 0, \qquad a \ne 0 \
\end{equation}

Его решение. \\
$D$ - дискриминант

\begin{equation}
  \label{eqn:discriminant}
  D = b^2 - 4ac
\end{equation}

\begin{equation}
  \label{eqn:qe_roots}
  x_{1,2} = \frac{-b \pm \sqrt{D}}{2a}
\end{equation}

Доказательство

Составим числовой пример. Рассмотрим выражение $x^2 + 10x$. При $x=2$ получаем $2^2 + 10 \cdot 2 = 24$. Таким образом, получаем следующее уравнение со знанием корня:

\begin{equation*}
  x^2 + 10x - 24 = 0, \qquad x = 2
\end{equation*}

Для его решения выделим полный квадрат. Т.е. сведём это уравнение к виду $(x + p)^2 = h$, где $p$ и $h$ какие-то числа. Такое уравнение уже можно решить с помощью квадратного корня. Для выделения полного квадрата вспомним формулу квадрата суммы:

\begin{equation}
  \left( a + b \right)^2 = a^2 + 2ab + b^2
\end{equation}

Используем символы $x$ и $p$:

\begin{equation*}
  \left( x + p \right)^2 = x^2 + 2xp + p^2
\end{equation*}

Пусть,\\
$ 10x = 2xp $ \\
$ 10 = 2p   $ \\
$ p = 5     $ \\
$ \left( x + 5 \right)^2 = x^2 + 10x + 25 $ \\
$ x^2 + 10x = \left( x + 5 \right)^2 - 25 $ \\
$ x^2 + 10x - 24 = \left( x + 5 \right)^2 - 25 - 24$ \\
$ x^2 + 10x - 24 = \left( x + 5 \right)^2 - 49$ \\
Т.е. уравнение \\
$ x^2 + 10x - 24 = 0 $ \\
Эквивалентно уравнению \\
$ \left( x + 5 \right)^2 - 49 = 0 $ \\
Решим его: \\
$ \left( x + 5 \right)^2 = 49 $ \\
$ \sqrt{\left( x + 5 \right)^2} = \sqrt{49} $ \\
\\
$ 7^2 = 49$ и ещё $ (-7)^2 = 49 $ \\
\\
Таким образом получаем 2 верных равенства: \\
$x + 5 = 7$ \qquad или \qquad $x + 5 = -7$ \\
От куда кроме решения $x=2$ так же видим новое решение $x=-12$. Действительно:\\
$(-12)^2 + 10 \cdot (-12) - 24 = 144 - 120 -24 = 0$.\\
То есть: \\
$ x^2 + 10x - 24 = 0 $, \\
$ x_1 = 2 $, \\
$ x_2 = -12 $

Теперь рассмотрим общее решение:\\
$ ax^2 + bx + c = 0, \qquad a \ne 0 $ \qquad :a \\
\begin{equation}
  \label{eqn:qe_reduced}
  x^2 + \frac{b}{a}x + \frac{c}{a} = 0
\end{equation}
Такое квадратное уравнение называется \textbf{приведенным} ($ a=1 $)\\
$ (x + p)^2 = x^2 + 2xp + p^2 $ \\
Пусть $ 2px = \dfrac{b}{a}x $ \\
$ 2p = \dfrac{b}{a} $ \\
$ p = \dfrac{b}{2a} $ \\
$ \left(x + \dfrac{b}{2a}\right)^2 = x^2 + \dfrac{b}{a}x + \dfrac{b^2}{4a^2} $ \\
$ x^2 + \dfrac{b}{a}x = \left( x + \dfrac{b}{2a} \right)^2 - \dfrac{b^2}{4a^2} $ \\
Далее подставляем в уравнение (\ref{eqn:qe_reduced}): \\
$ \left( x + \dfrac{b}{2a} \right)^2 - \dfrac{b^2}{4a^2} + \dfrac{c}{a} = 0 $ \\
$ \left( x + \dfrac{b}{2a} \right)^2 = \dfrac{b^2}{4a^2} - \dfrac{c}{a}$ \\
Приведем справа к общему знаменателю: \\
$ \left( x + \dfrac{b}{2a} \right)^2 = \dfrac{b^2 - 4ac}{4a^2} $ \\
Извлечем квадратный корень: \\
$ \sqrt{\left( x + \dfrac{b}{2a} \right)^2} = \sqrt{\dfrac{b^2 - 4ac}{4a^2}} $ \\
Получаем 2 верных равенства: \\
$ x + \dfrac{b}{2a} = -\dfrac{\sqrt{b^2 - 4ac}}{2a} $, \qquad или \\
$ x + \dfrac{b}{2a} = +\dfrac{\sqrt{b^2 - 4ac}}{2a} $ \\
Выражение под квадратным корнем и есть дискриминант (\ref{eqn:discriminant}) \\
$ x + \dfrac{b}{2a} = -\dfrac{\sqrt{D}}{2a} $, \qquad или \\
$ x + \dfrac{b}{2a} = +\dfrac{\sqrt{D}}{2a} $ \\
------------------------------------------------\\
$ x_1 = \dfrac{-b - \sqrt{D}}{2a} $, \qquad или \\
$ x_2 = \dfrac{-b + \sqrt{D}}{2a} $ \\
Что и требовалось доказать (\ref{eqn:qe_roots})

О термине дискриминант. От латинского \textbf{discrimino} - разбираю, различаю. От сюда же дискриминация.

В квадратных уравнениях возможны следующие случаи:\\
1) $ D < 0 $ - уравнение не имеет корней в области вещественных чисел;  \\
2) $ D = 0 $ - уравнение имеет два одинаковых корня $ x_1=x_2=-\dfrac{b}{2a}; $ \\
3) $ D > 0 $ - уравнение имеет два разных корня в области вещественных чисел (\ref{eqn:qe_roots}).

Т.е. сравнивая значение дискриминанта $D$, можно разобрать, различить некоторую информацию о корнях этого уравнения.

\clearpage

\textbf{Источники}

EN

1. \href{https://www.gnu.org/software/make/}{https://www.gnu.org/software/make/}  \\
2. Robert Mecklenburg. Managing Projects with GNU Make

RU

3.\href{https://rus-linux.net/nlib.php?name=/MyLDP/algol/gnu_make/gnu_make_3-79_russian_manual.html}{https://rus-linux.net/nlib.php?name=/MyLDP/algol/gnu\_make/gnu\_make\_3-79\_russian\_manual.html}

\href{http://linux.yaroslavl.ru/docs/prog/gnu_make_3-79_russian_manual.html}{http://linux.yaroslavl.ru/docs/prog/gnu\_make\_3-79\_russian\_manual.html}

\clearpage

Далее идут черновики по latex'у :)

% pgfmanual, page 142

\begin{tikzpicture}[fill=blue!20]
	\draw [help lines] (0, 0) grid (6, 3);
	\draw (0, 0) -- (6, 1);

\end{tikzpicture}


% pgfmanual, page 35

We are working on
\begin{tikzpicture}
	\draw (-1.5,0) -- (1.5,0);
	\draw (0,-1.5) -- (0,1.5);
	\draw (0, 0) circle [radius=1];
\end{tikzpicture}.

\begin{tikzpicture}
	\draw [help lines] (-2, -2) grid (2, 3);
	\path ( 0,2) node [shape=circle,draw] {}
	( 0,1) node [shape=circle,draw] {}
	( 0,0) node [shape=circle,draw] {}
	( 1,1) node [shape=rectangle,draw] {}
	(-1,1) node [shape=rectangle,draw] {};
\end{tikzpicture}

\begin{tikzpicture}
	\draw [help lines] (-2, -2) grid (2, 3);
	\node (A)  at ( 0,2) [shape=circle,draw] {A};
	\node (B) at ( 0,1) [shape=circle,draw] {B};
	\node (C) at ( 0,0) [shape=circle,draw] {C};
	\node (D) at ( 1,1) [shape=rectangle,draw] {D};
	\node (E) at (-1,1) [shape=rectangle,draw] {E};

	\draw[->] (E.east) -- (B.west);
\end{tikzpicture}

\end{document}}
