% Преамбула

% oneside - одностороняя печать (есть ещё twoside)
% final   - финальная версия документа (есть ещё draft)
% 14pt    - размер шрифта 14-й кегль
\documentclass[oneside,final,12pt]{book}

\usepackage[T2A]{fontenc}

% В книге говориться, что можно использовать koi8-r (для Linux)
% и cp1251 для Windows
% Я выбрал utf8. .tex файлы сохраняю в этой кодировке. Работает на Linux и Windows.
\usepackage[utf8]{inputenc}

% Пакет babel адаптирует LATEX к работе с русским языком
\usepackage[russianb]{babel}

% Настраивание размера полосы набора BEGIN======================================
\usepackage{vmargin}
\setpapersize{A4}

% Значение параметров:
% 1     - левое поле
% 2     - верхнее поле
% 3     - правое поле
% 4     - нижнее поле
% 5,6,7 - для управления верхним и нижним колонтитулами
% 8     - расположение по вертикали номреа страницы. Расстояние между нижним краем нижней строки и нижним краем номера страницы
%\setmarginsrb{2cm}{0cm}{1cm}{1.5cm}{0pt}{0mm}{0pt}{13mm}
% Настраивание размера полосы набора END========================================

\usepackage{epigraph}

% Красная строка
\usepackage{indentfirst}

% Оглавление и ссылки как гиперсылки
% Это для удобства чтения в электронном виде
\usepackage[unicode]{hyperref}

% Предотвращшение залезания строк на поля, даже если для этого требуется заполнить строку недопустимо длинными пробелами
\sloppy

% Выравнивание подписей по левому краю
\usepackage{caption}
\captionsetup{justification=raggedright, singlelinecheck=false}

% Для таблиц
\usepackage{makecell}

% Для вставки заранее подготовленных рисунков
\usepackage{graphicx}

% Листинги программ
\usepackage{verbatim}
\usepackage{moreverb}
\usepackage{listings}

\lstdefinestyle{texstyle}
{
    breaklines=true,
    frame=single,
    inputencoding=utf8,
    language=TeX,
    numbers=left,
    % для нормального отображения кириллицы в комментариях
    extendedchars=\true,
    keepspaces=true
}

\lstset{style=texstyle}

% Математические формулы
\usepackage{amsmath}
\usepackage{amsfonts}

% Для устранения ворнингов про \circle, \oval...
\usepackage{pict2e}

% Определение (sbdef - serge beloff definition)
\newcommand{\sbdef}[1]{\textbf{#1}}
